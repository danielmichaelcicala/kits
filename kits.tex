\documentclass{article}
\usepackage{amsthm,tikz}
\usetikzlibrary{cd}

%---------------------------------------------------------------	
\newtheorem*{theorem*}{Theorem}
\newtheorem{definition}{Definition}[section]
\newtheorem{proposition}[definition]{Proposition}   
\newtheorem{theorem}[definition]{Theorem}
\newtheorem{lemma}[definition]{Lemma}   
\newtheorem{corollary}[definition]{Corollary}   

\theoremstyle{remark}
\newtheorem{example}[definition]{Example}
\newtheorem*{example*}{Example}
\newtheorem{remark}[definition]{Remark}

%---------------------------------------------------------------

\title{Kits, or whatever they'll come to be called}
\author{}
\date{}

%---------------------------------------------------------------

\begin{document}
\maketitle

\section{Introduction}

Kits are an algebraic structure that models networking.

\section{Definition}

A \emph{typed finite set} (with respect to some set of types $T$) is a finite
set $B$ together with a function $B \to T$.

A \emph{typed equivalence relation} between typed finite sets $B$ and $C$ is a
function $e\colon B+C \to P$ and a map $P \to T$ such that
\[
\begin{tikzcd}
B+C \ar[rr, "e"] \ar[dr] && P \ar[dl]\\
& T
\end{tikzcd}
\]
commutes. We can compose equivalence relations by taking the finest equivalence
relation that is coarser than the factors.

\begin{definition}
A \emph{kit} $K$ is
\begin{itemize}
\item a collection of \emph{types} $T$;
\item for all typed finite sets $B$ over $T$ (called a \emph{boundary}), a set
of \emph{networks} $K_B$;
\item for all boundaries $B$, $C$, a map $K_B\times K_C \to K_{B+C}$;
\item for all boundaries $B$, $C$ and all typed equivalence relations $e\colon B
\to C$, a function $K_e\colon K_B \to K_C$ (called \emph{composition along
$e$});
\end{itemize}
obeying the axioms
\end{definition}

A morphism of kits $K \to K'$ is a function on types $f\colon T \to T'$ for all
boundaries $B$ in $T$ a function $K_B \to K_{f(B)}$ obeying some naturality
requirements.

\section{First examples}
\subsection{The kit of open graphs}
Networks are open graphs

\subsection{The kit of colimits}
Networks are wide cospans.

\subsection{The kit of black boxed colimits}
Networks are wide corelations.

\section{The category of kits}

\subsection{Limits}

\begin{example}
Terminal kit
\end{example}

\begin{example}
Products of kits
\end{example}


\subsection{Colimits}
\begin{example}
Initial kit
\end{example}
\begin{example}
Coproducts of kits
\end{example}

\section{Presenting kits}
\subsection{Signatures}
A signature for a kit is a collection of types and a set of networks for each
type.
\subsection{The free kit on a signature}
This defines an adjunction between signatures and kits.


\section{Hypergraph categories}

This are equivalent to kits.
\subsection{Frobenius monoids}

\section{Constructing kits using decorated corelations}
Decorated corelations freely constructs kits given some kit structure.
\subsection{Decorated cospans}

\subsection{Decorated corelations}

\section{Kits as operad algebras}
Kits are algebras for the corelation operad.

\end{document}
